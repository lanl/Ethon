\documentclass{article}

\usepackage{amsmath}
\usepackage{bm}
\usepackage{natbib}
\usepackage{hyperref}
\usepackage{cleveref}

% Turn off ChkTeX warning about \( \) instead of $ $
% chktex-file 46

\begin{document}

\author{Jonas Lippuner}
\title{Ethon hydrodynamics notes}

\maketitle

Ethon is a free and open source software framework to test and develop hydrodynamics methods
with block-based  adaptive mesh refinement on various parallel computing architectures. It
is available at \url{https://github.com/lanl/Ethon} and described in detail in \cite{lippuner:21}.
These notes are distributed together with the Ethon source code and they describe
the hydrodynamics physics currently implemented in Ethon.

Throughout these notes, we will use index notation for vectors and derivatives. The vector $\vec x$
will be denoted as $x^i$, where $i$ ranges from 1 to 3. Furthermore, we denote time and spacial
derivatives as follows.
\begin{align}
  \frac{\partial}{\partial t} = \partial_t
  \qquad \text{and} \qquad
  \frac{\partial}{\partial x_i} = \partial_i.
\end{align}
Note that we are in flat space-time, hence covariant and contravariant indices are the same.
Repeated indices are summed over. For example, the gradient of a scalar and divergence of a vector
field are written as
\begin{align}
  \label{eq:examples}
  \nabla f = \partial_i f
  \qquad \text{and} \qquad
  \nabla \cdot \vec u = \partial_i u^i.
\end{align}

\section{Conserved and primitive variables}

We define the following \textbf{primitive} variables:
\begin{itemize}
  \item Fluid density $\rho$
  \item Fluid velocity $u^i$
  \item Fluid internal specific energy $e$
  \item Fluid stress tensor $P^{ij}$
\end{itemize}
For now we only consider isotropic stresses and so we write $P^{ij} = p\delta^{ij}$, where $p$ is
the usual fluid pressure and $\delta^{ij}$ is the Kronecker delta.

The \textbf{conserved} variables are the ones that are actually evolved and they are:
\begin{itemize}
  \item Mass density $\rho$
  \item Momentum density $\mu^i = \rho u^i$
  \item Total energy density $\epsilon = \rho e + \frac{1}{2}\rho u_i u^i$
\end{itemize}
Note that the total energy density is the sum of the internal energy density and kinetic energy
density.

\section{Compressible Euler equations}

We write the conservation equations for the conserved variables as follows.
\begin{align}
  \label{eq:mass_cons}
  \partial_t \rho + \partial_j(\rho u^j)  & = 0 \\
  \label{eq:mom_cons}
  \partial_t \mu^i + \partial_j(\rho u^i u^j + P^{ij}) & = \rho g^i \\
  \label{eq:ener_cons}
  \partial_t \epsilon + \partial_j(\epsilon u^j + u_i P^{ij}) & = \rho u_i g^i,
\end{align}
where $g^i$ is the external acceleration due to gravity. Note that we can write the above also in
terms of the time derivatives of the primitive variables.
The mass conservation equation \labelcref{eq:mass_cons} remains the same. The momentum conservation
equation \labelcref{eq:mom_cons} becomes
\begin{align}
  \label{eq:dtu}
  & \partial_t(\rho u^i)
  + \partial_j(\rho u^i u^j + P^{ij})
  = \rho g^i
  \nonumber \\
  \Leftrightarrow\ \
  & u^i\left(\partial_t + \partial_j(\rho u^j)\right)
  + \rho \partial_t u^i
  + \rho u^j \partial_j u^i
  + \partial^j p \delta^{ij}
  = \rho g^i
  \nonumber \\
  \Leftrightarrow\ \
  & \partial_t u^i
  + u^j \partial_j u^i
  + \frac{\partial^i p}{\rho}
  = g^i,
\end{align}
where we used \cref{eq:mass_cons}, $\mu^i = \rho u^i$, and $P^{ij} = p\delta^{ij}$. Finally, for the
energy conservation equation \labelcref{eq:ener_cons} we find
\begin{align}
  & \partial_t\left(\rho e + \frac{1}{2}\rho u_i u^i\right)
  + \partial_j\left(\rho e u^j + \frac{1}{2}\rho u_i u^i u^j + u_i P^{ij}\right)
  = \rho u_i g^i
  \nonumber \\
  \Leftrightarrow\ \
  & e\partial_t\rho + \rho \partial_t e + \frac{1}{2}u_i u^i\partial_t \rho + \rho u_i\partial_t u^i
  \nonumber \\
  & {} + e\partial_j(\rho u^j) + \rho u^j \partial_j e + \frac{1}{2}u_i u^i \partial_j(\rho u^j)
    + \rho u^j u_i\partial_j u^i
  \nonumber \\
  & {} + u_i \partial^i p + p\partial^i u_i
  = \rho u_i g^i
  \nonumber \\
  \Leftrightarrow\ \
  & e\left(\partial_t \rho + \partial_j(\rho u^j)\right)
  \nonumber \\
  & {} + \frac{1}{2}u_i u^i\left(\partial_t \rho + \partial_j(\rho u^j)\right)
  \nonumber \\
  & {} + \rho u_i \left(\partial_t u^i + u^j\partial_j u^i + \frac{\partial^i p}{\rho} - g^i\right)
  \nonumber \\
  & {} + \rho \partial_t e + \rho u^j \partial_j e + p \partial^i u_i = 0
  \nonumber \\
  \Leftrightarrow\ \
  & \partial_t e + u^i\partial_i e + \frac{p}{\rho}\partial^i u_i = 0,
\end{align}
where we used \cref{eq:mass_cons,eq:dtu}, $\epsilon = \rho e + \frac{1}{2}\rho u_i u^i$, and
$P^{ij} = p\delta^{ij}$.

If shocks are present, non-conservative methods do not converge to the correct solution, so one must
use conservative methods in those cases (which do converge to the correct solution, if they converge
at all), see \citet[][\S 5.3]{toro}.

\newcommand{\vc}[1]{\mathrm{\bf #1}}

For notational simplicity, we shall write the conservative form of the compressible Euler equations
\labelcref{eq:mass_cons,eq:mom_cons,eq:ener_cons} as follows
\begin{align}
  \label{eq:euler_vec}
  \partial_t \vc{U} + \partial_j \vc{F}^j(\vc U) = \vc{S}(\vc U),
\end{align}
where the state vector $\vc U$, the flux tensor $\vc F^j(\vc U)$, and the source term
$\vc S(\vc U)$ are given by
\begin{align}
  \label{eq:system_vecs}
  \vc U =
    \begin{bmatrix}
      \rho  \\
      \mu^1 \\
      \mu^2 \\
      \mu^3 \\
      \epsilon
    \end{bmatrix}, \qquad
  \vc F^j(\vc U) =
    \begin{bmatrix}
      \rho u^j \\
      \rho u^1 u^j + P^{1j} \\
      \rho u^2 u^j + P^{2j} \\
      \rho u^3 u^j + P^{3j} \\
      \epsilon u^j + u_i P^{ij}
    \end{bmatrix}, \qquad
  \vc S(\vc U) =
    \begin{bmatrix}
      0 \\
      \rho g^1 \\
      \rho g^2 \\
      \rho g^3 \\
      \rho u_i g^i
    \end{bmatrix}.
\end{align}
Note that to evaluate $\vc F^j(\vc U)$ we need an additional Equation of State (EOS) that allows us
to recover the primitive variables from the conservative ones, which are contained in $\vc U$. Note
that all of the above quantities are functions of time and space.


\section{Finite-volume scheme}

To discretize the non-linear, partial-differential Euler equations \labelcref{eq:mass_cons,%
eq:mom_cons,eq:ener_cons} we introduce a Cartesian mesh. For simplicity, we initially consider a
uniform mesh with cells of size $\Delta x^i$, where the sizes in the three dimensions can vary, in
principle (in practice they are usually the same, though). Furthermore, we consider a cell-centered
method where we have grid points at the centers of the cells and the cell-averaged quantities are
stored at those grid points. Additionally, we initially restrict ourselves to the case without
source terms, i.e.\ $\vc S = \vc 0$ and so we have the system
\begin{align}
  \label{eq:sys}
  \partial_t \vc U + \partial_j \vc F^j(\vc U) = \vc 0.
\end{align}

\newcommand{\dv}{\,\mathrm{d}V}
\newcommand{\ds}{\,\mathrm{d}S}
\newcommand{\dt}{\,\mathrm{d}t}

Now consider a cell with index $\alpha$ and volume
\begin{align}
  V_\alpha = \Delta x^1 \Delta x^2 \Delta x^3.
\end{align}
We define the cell average of the state vector in cell $\alpha$ as
\begin{align}
  \label{eq:cell-average}
  \bar{\vc U}_\alpha(t) = \frac{1}{V_\alpha} \int_{V_\alpha} \vc U(x^i,t) \dv.
\end{align}
Since $\partial_t \vc U = -\partial_j\vc F^j(\vc U)$, we have
\begin{align}
  \vc U(x^i,t_2) = \vc U(x^i, t_1) - \int_{t_1}^{t_2} \partial_j \vc F^j(\vc U(x^i,t)) \dt,
\end{align}
and so
\begin{align}
  \label{eq:avg-update}
  \bar{\vc U}_\alpha(t_2) &= \frac{1}{V_\alpha}\int_{V_\alpha} \left( \vc U(x^i,t_1)
    - \int_{t_1}^{t^2} \partial_j \vc F^j(\vc U(x^i, t))\dt \right)\dv
  \nonumber \\
  &= \bar{\vc U}_\alpha(t_1) - \frac{1}{V_\alpha}\int_{t_1}^{t_2}\int_{\partial V_\alpha}
    n_j \vc F^j(\vc U(x^i, t))\ds\dt,
\end{align}
where we assumed that the flux $\vc F$ is well behaved so that we can change the order of
integration, and
we used the divergence theorem to replace the volume integral of the divergence of $\vc F$ with the
surface integral over the cell $\alpha$, where $n_j$ is the normal vector of the cell surface. Note
that \cref{eq:avg-update} is exact since we have made no approximations yet.

The surface integral in \cref{eq:avg-update} has six components, namely the integral of the flux
across the six faces of the cell. For cell $\alpha$ we define the following quantities
\begin{align}
  \label{eq:cell-def}
  x_\alpha^i &= \text{cell center location}, \nonumber \\
  x_{\alpha,L}^i &= \text{coordinates of the lower cell faces, and} \nonumber \\
  x_{\alpha,U}^i &= \text{coordinates of the upper cell faces.}
\end{align}
Note that
\begin{align}
  x_{\alpha,U}^i - x_{\alpha,L}^i = \Delta x^i \qquad \text{and} \qquad
  x_\alpha^i = \frac{x_{\alpha,L}^i + x_{\alpha,U}^i}{2}.
\end{align}
We can now write \cref{eq:avg-update} as
\begin{align}
  \label{eq:finite-volume}
  \bar{\vc U}_\alpha(t_2) = \bar{\vc U}_\alpha(t_1) + \frac{\Delta t}{\Delta x^i}
          \left(\bar{\vc F}^i_{\alpha,L} - \bar{\vc F}^i_{\alpha,U}\right),
\end{align}
where $\Delta t$ = $t_2 - t_1$, and $\bar{\vc F}^i_{\alpha,L}$ and
$\bar{\vc F}^i_{\alpha,U}$ are the average fluxes through the lower and upper faces in the
$i$-direction. Their exact form is
\begin{align}
  \label{eq:fluxes}
  \bar{\vc F}^i_{\alpha,L} = \frac{\Delta x^i}{\Delta t V_\alpha}\int_{t_1}^{t_2}
      \int_{V_\alpha} \vc F^i(\vc U(x^1, x^2, x^3, t)) \delta(x^i - x^i_L) \dv \dt
      \quad \text{(no sum over $i$)},
\end{align}
where the Dirac-Delta function reduces the integral over the cell volume to an integral over the
lower cell face in direction $i$ with coordinate $x^i_L$. The expression for
$\bar{\vc F}^i_{\alpha,U}$ is analogous, since we have already taken care of the sign of the
normal vector of the lower and upper faces in \cref{eq:finite-volume}.

Obviously, we cannot evaluate \cref{eq:fluxes} directly, because that would require knowledge of the
solution $\vc U$ at all times and all points in space. One of the core ingredients of a finite
volume method is thus to prescribe some numeric approximations for $\bar{\vc F}^i_{\alpha,L}$
and $\bar{\vc F}^i_{\alpha,U}$ that can be readily evaluated.

\subsection{Godunov scheme}

The Godunov scheme is one of the simplest finite volume schemes and it is first-order
accurate. We compute the fluxes by solving the Riemann Problem between the two cells, i.e.
\begin{align}
\bar{\vc{F}}^j_{\alpha,L} = \text{RP}(\bar{\vc U}_{\alpha-1}, \bar{\vc U}_\alpha), \\
\bar{\vc{F}}^j_{\alpha,U} = \text{RP}(\bar{\vc U}_\alpha, \bar{\vc U}_{\alpha+1}),
\end{align}
where $\bar{\vc U}_{\alpha-1}$ and $\bar{\vc U}_{\alpha+1}$ are the average states of the cells
adjacent to cell $\alpha$ in the lower and upper direction in dimension $j$, respectively,
and we use RP to denote the solution to the Riemann Problem. We can use use these fluxes in
\cref{eq:finite-volume} to update the state in cell $\alpha$.


\subsection{MUSCL-Hancock scheme}

The MUSCL-Hancock scheme is a popular finite volume scheme that is second-order accurate in time and
space. MUSCL stands for Monotone Upstream-centred Scheme for Conservation Laws. For example, FLASH
uses MUSCL-Hancock for its dimensionally unsplit hydro solver. See
\cite[\S13.4, \S14.4, \S16.5]{toro} for more details.

The central idea in the MUSCL-Hancock scheme is to replace the constant cell averages by a linearly
varying quantity within the cell. Considering only one dimension for a moment, we define the
conserved quantity $\vc U_\alpha(x)$ in the cell $\alpha$ as
\begin{align}
  \label{eq:ux}
  \vc U_\alpha(x) = \bar{\vc U}_\alpha + \frac{x - x_\alpha}{\Delta x} \vc\Delta_\alpha
  \qquad\text{for } x\in [x_{\alpha,L}, x_{\alpha,U}],
\end{align}
where $\vc\Delta_\alpha$ is a suitably chosen slope vector (of the five conserved quantities)
of the solution inside the cell (see \cref{sec:slope} for details on how to construct the slope
vector).
The above is at a fixed time.
We now evaluate the above at the lower and upper boundaries
to get the extrapolated boundary values of $\vc U$:
\begin{align}
  \label{eq:extrapolated-boundary-values-1d}
  \vc U^L_\alpha = \vc U_\alpha(x_{\alpha,L}) =
    \bar{\vc U}_\alpha - \frac{\vc\Delta_\alpha}{2}
  \quad \text{and} \quad
  \vc U^U_\alpha = \vc U_\alpha(x_{\alpha,U}) =
    \bar{\vc U}_\alpha + \frac{\vc\Delta_\alpha}{2}.
\end{align}
We now evolve the boundary extrapolated conserved quantities for a half timestep $\Delta t/2$
according to \cref{eq:finite-volume} by evaluating $\vc F$ for the boundary extrapolated values
$\vc U^L_\alpha$ and $\vc U^U_\alpha$ and using these as the approximate fluxes. Hence
we get
\begin{align}
  \label{eq:half-time-step-1d}
  \hat{\vc U}^L_\alpha &= \vc U^L_\alpha + \frac{\Delta t}{2\Delta x}
      \left[\vc F(\vc U^L_\alpha) - \vc F(\vc U^U_\alpha)\right], \nonumber \\
  \hat{\vc U}^U_\alpha &= \vc U^U_\alpha + \frac{\Delta t}{2\Delta x}
      \left[\vc F(\vc U^L_\alpha) - \vc F(\vc U^U_\alpha)\right],
\end{align}
where all the right-hand side terms are evaluated at the current time $t_1$. Finally, to obtain the
actual approximate fluxes at the cell boundaries, we solve the one-dimensional local Riemann problem
at the cell boundaries using the evolved boundary extrapolated quantities. A thorough discussion of
the Riemann Problem is beyond the scope of these notes. Simply put, the solution to Riemann Problem
roughly provides the flux across a discontinuity namely a cell face. For more details, see \cite{toro}.
The Riemann Problem
requires left and right initial states $\vc U_L$ and $\vc U_R$ and produces a similarity solution
$\vc U(x/t)$, which we will denote by
\begin{align}
  \label{eq:RP-notation}
  \vc U(x/t) \equiv \text{RP}(\vc U_L, \vc U_R).
\end{align}
At the lower cell boundary of cell $\alpha$, the left initial state for the Riemann problem is
$\hat{\vc U}_{\alpha-1}^U$ and the right initial state is $\hat{\vc U}_\alpha^L$, where $\alpha -1$
denotes the cell to the left of cell $\alpha$ in the $x$-direction. The initial states for the
Riemann Problem at the upper cell boundary of cell $\alpha$ are analogous. Hence we have
\begin{align}
  \label{eq:RPs-1d}
  \vc U_\alpha^L(x/t) &= \text{RP}(\hat{\vc U}_{\alpha-1}^U, \hat{\vc U}_\alpha^L), \nonumber \\
  \vc U_\alpha^U(x/t) &= \text{RP}(\hat{\vc U}_\alpha^U, \hat{\vc U}_{\alpha+1}^L),
\end{align}
where $\alpha+1$ is the cell to the right of cell $\alpha$ in the $x$-direction. We now use the
Riemann similarity solutions evaluated at $x/t = 0$ to get the final approximate fluxes for
\cref{eq:finite-volume}, thus we have
\begin{align}
  \label{eq:fluxes-1d}
  \bar{\vc F}_{\alpha,L} = \vc F(\vc U_\alpha^L(0)) \qquad \text{and} \qquad
  \bar{\vc F}_{\alpha,U} = \vc F(\vc U_\alpha^U(0)).
\end{align}
Note that in principle, we could solve the Generalized Riemann Problem where the left and right
states are not constant. We could use a linear form similar to \cref{eq:ux} to describe the left and
right states for the Riemann Problems at the boundaries. However, the Generalized Riemann Problem is
exceedingly difficult and not typically employed \cite[\S13.4.1]{toro}.

Generalizing the above to three dimensions is straight forward. We still solve one-dimensional,
piece-wise constant Riemann Problems at the cell boundaries (now there are six instead of two), but
for the time evolution in \cref{eq:half-time-step-1d} we incorporate the fluxes from all six faces.
We now have a different slope vector (see \cref{sec:slope}) in each dimension and we'll denote the
component in dimension $i$ by $\vc \Delta_\alpha^i$. We have six boundary extrapolated values,
namely
\begin{align}
  \label{eq:boundary-values-3d}
  \vc U_\alpha^{L,i} = \bar{\vc U}_\alpha - \frac{\vc \Delta_\alpha^i}{2}
  \qquad \text{and} \qquad
  \vc U_\alpha^{U,i} = \bar{\vc U}_\alpha + \frac{\vc \Delta_\alpha^i}{2},
\end{align}
which are evolved forward in time by $\Delta t/2$ to yield
\begin{align}
  \label{eq:evolved-boundary-values-3d}
  \hat{\vc U}_\alpha^{[L|U],i} = \vc U_\alpha^{[L|U],i} + \frac{\Delta t}{2\Delta x^j}
      \left[\vc F^j(\vc U_\alpha^{L,j}) - \vc F^j(\vc U_\alpha^{U,j})\right].
\end{align}
Now we solve the Riemann Problem at all six faces using one evolved boundary extrapolated value from
this cell and one from the adjacent cell, just like in \cref{eq:RPs-1d}. This yields a similarity
solution at each face, which we plug into the flux function $\vc F$ to get the approximate flux
$\bar{\vc F}^i_{\alpha,[L|U]}$
through that face. The cell averaged conserved quantity $\bar{\vc U}_\alpha$ is then updated
according to \cref{eq:finite-volume}.


\subsection{Slope construction and limiter methods}\label{sec:slope}

\newcommand{\du}{\vc\Delta_\alpha^\text{up}}
\newcommand{\dd}{\vc\Delta_\alpha^\text{down}}
\newcommand{\dc}{\vc\Delta_\alpha^\text{cent}}
\newcommand{\db}{\bar{\vc\Delta}_\alpha}

We will restrict ourselves again to one dimension in this section, since it drastically simplifies
notation. The components of the slope vector in the different dimensions are all determined
independently from each other, so generalizing this discussion to multiple dimension is trivial.
We use $\alpha+1$ to denote the neighboring cell in the positive direction and $\alpha-1$ the
neighbor in the negative direction. We follow \cite[\S6]{leveque}. A general form of the slope is
\begin{align}
  \label{eq:slope_vector}
  \vc \Delta_\alpha =
      \frac{1}{2}(1+\omega) \left(\bar{\vc U}_{\alpha} - \bar{\vc U}_{\alpha-1}\right)
    + \frac{1}{2}(1-\omega) \left(\bar{\vc U}_{\alpha+1} - \bar{\vc U}_{\alpha}\right),
\end{align}
where $\omega$ is a free parameter in the interval $[-1,1]$. Three common choices for $\omega$ are
\begin{align}
  \label{eq:example_slopes}
  \text{Upwind ($\omega = 1$):} \quad
    & \du = \bar{\vc U}_{\alpha} - \bar{\vc U}_{\alpha-1} &
    & \text{(Beam--Warming).}  \\
  \text{Centered ($\omega = 0$):} \quad
    & \dc = \frac{\bar{\vc U}_{\alpha+1} - \bar{\vc U}_{\alpha-1}}{2} &
    & \text{(Fromm),}  \\
  \text{Downwind ($\omega = -1$):} \quad
    & \dd = \bar{\vc U}_{\alpha+1} - \bar{\vc U}_{\alpha} &
    & \text{(Lax--Wendroff),}
\end{align}

Using any of these slopes leads to a second-order method. Unfortunately, with the increased accuracy
also come spurious oscillations near discontinuities. To avoid these discontinuities, the slope
computed above needs to be limited in a way that ensures that the total variation of the solution
does not increase. The resulting scheme is said to be total variation diminishing (TVD). Let
$\db$ denote the limited slope and define the minmod and maxmod functions as
\begin{align}
  \label{eq:minmod-maxmod}
  \text{minmod}(a,b) = \left\{
    \begin{array}{ll}
      0 & \text{if } ab \leq 0, \\
      a & \text{if } |a| \leq |b| \text{ and } ab > 0, \\
      b & \text{if } |b| \leq |a| \text{ and } ab > 0.
    \end{array}\right. \\[6pt]
    \text{maxmod}(a,b) = \left\{
    \begin{array}{ll}
      0 & \text{if } ab \leq 0, \\
      a & \text{if } |a| \geq |b| \text{ and } ab > 0, \\
      b & \text{if } |b| \geq |a| \text{ and } ab > 0.
    \end{array}\right.
\end{align}

Some popular limiters are the following.
\begin{align}
  \label{eq:slope-limiters}
  \text{minbee:}   \quad & \db = \text{minmod}\left(\du, \dd\right) \\[6pt]
  \text{superbee:} \quad & \db = \text{maxmod}\big[\text{minmod}\left(\du, 2\dd\right), \nonumber \\
      & \hphantom{\db = \text{maxmod}\big[} \text{minmod}\left(2\du,\dd\right)\big] \\[6pt]
  \text{MC:} \quad       & \db = \left\{ \begin{array}{ll}
    \phantom{-}0 & \text{if } \du\dd \leq 0, \\
    \phantom{-}\text{min}(2|\du|, 2|\dd|, |\dc|) & \text{if } \dc > 0, \\
    -\text{min}(2|\du|, 2|\dd|, |\dc|) & \text{if } \dc < 0.
  \end{array} \right. \\[6pt]
  \text{van Leer:} \quad & \db = \left\{ \begin{array}{ll}
    0 & \text{if } \du\dd \leq 0, \\
    \dfrac{2\du\dd}{\du+\dd} & \text{otherwise.}
  \end{array} \right.
\end{align}
Note that the minbee limiter is also called minmod limiter.

\section{$\gamma$-Law Equation of State}

Let $\gamma$ be the adiabatic index (ratio constant-pressure heat capacity to constant-volume heat
capacity). The pressure is given by
\begin{align}
  \label{eq:gamma-Law}
  P = (\gamma-1)\rho e,
\end{align}
where $\rho$ is the mass density and $e$ is the specific internal energy. We have $\gamma = 5/3$ for
monoatomic ideal gases and $\gamma = 7/5$ for diatomic ideal gases. From the ideal gas law we
also have
\begin{align}
  \label{eq:ideal-gas}
  PV = Nk_BT \Rightarrow P = \frac{N}{V}k_BT = nk_BT = \frac{\rho}{m}k_BT,
\end{align}
where $V$ is the volume, $N$ is the number of particles, $n = N/V$ is the number density, and $m$ is
the mass of a particle. Equating the above gives
\begin{align}
  \label{eq:gamma-law_e}
  e = \frac{1}{\gamma-1}\frac{k_BT}{m}.
\end{align}
For an isentropic (constant entropy, i.e.\ adiabatic and reversible) process, one can derive
\begin{align}
  \label{eq:isentropic}
  P = \alpha \rho^\gamma,
\end{align}
for some constant $\gamma$. The sound speed can now be computed as
\begin{align}
  \label{eq:sound-speed}
  c_s^2 = \left.\frac{dP}{d\rho}\right|_S = \gamma \alpha \rho^{\gamma-1} = \gamma\frac{P}{\rho} =
  \gamma(\gamma-1)e.
\end{align}

\section*{Acknowledgements}

The development of Ethon was funded by the Laboratory Directed
Research and Development program of Los Alamos National Laboratory under
project number 20190519ECR. The development used LANL Institutional Computing
Program resources. LANL is operated by Triad National Security, LLC, for the
National Nuclear Security Administration of the U.S.DOE (Contract No.\
89233218CNA000001).

\bibliographystyle{apj}
\bibliography{notes.bib}

\end{document}
